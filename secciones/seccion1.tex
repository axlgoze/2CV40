\section{Título de la primera sección}
Contenido de la primer sección. Primer bibliografia:\cite{bib1}


\subsection{Título de la primera subsección}
Contenido de la primer subsección. 

\begin{center}
Algo centrado\footnote{nota al pie} %nota al pie.
\end{center}


\subsection{Título de la segunda subsección}
Contenido de la segunda subsección. La tabla 1 es un ejemplo de tabla.

\begin{table} [h]%ambiente para tablas.
    \begin{center}
        \begin{tabular}{ccc} %iconstruccion de tabla, centrado, centrado, centrado.
            \textbf{UNO} & \textbf{DOS} & \textbf{TRES} \\ \hline %textbf negritas, \\salto de linea.
            a & b & c \\
            d & e & f \\ \hline %crea horizontal line.
        \end{tabular}
    \end{center}
    \caption{Ejemplo de tabla. \label{tabla}}
\end{table}
\pagebreak
\subsection{Título de la tercera subsección}
Contenido de la segunda subsección. A continuación se muestra un ejemplo de enumeración:

\begin{enumerate}
    \item \textit{uno.}
    \item \underline{dos.}
    \item \textbf{tres.}
    \item cuatro.
\end{enumerate}
