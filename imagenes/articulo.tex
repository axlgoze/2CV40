% Articulo funge como main
\documentclass[letterpaper]{article} %documento tipo articulo
\usepackage[utf8]{inputenc} %permite uso de tildes
\usepackage{graphicx} %permite la incorporacion de figuras y graficos
\usepackage[spanish]{babel} %ambiente que maneja varios idiomas, en especifico usamos español
\usepackage{pdfpages} 

% Estructura fundamental del artiuclo
\title{Título del articulo} %titulo del articulo
\author{Axel Reyes} %autor
\date{08 junio 2020} %fecha fija

\begin{document}
    \maketitle %incluye estrucutra del titulo
    %\begin{abstract}
    Resumen del artículo.
\end{abstract}  aqui mandamos a llamar archivo donde se encuentra el resumen
    \renewcommand{\tablename}{Tabla} %Para cambiar nombre: cuadro por tabla

    % Secciones del artículo 
    %\input{}
    %\input{}
    %\input{}

    % Referencias
    %\input{}
\end{document}
