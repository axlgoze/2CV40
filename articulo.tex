% Articulo funge como main
\documentclass[letterpaper]{article} %documento tipo articulo
\usepackage[utf8]{inputenc} %permite uso de tildes
\usepackage{graphicx} %permite la incorporacion de figuras y graficos
\usepackage[spanish]{babel} %ambiente que maneja varios idiomas, en especifico usamos español
\usepackage{pdfpages} 
\usepackage{pifont} %paquete que incluye simbolos especiales.

% Estructura fundamental del articulo
\title{Título del articulo \LaTeX} %titulo del articulo
\author{Axel Reyes} %autor
\date{08 junio 2020} %fecha fija

\begin{document}
    \maketitle %incluye estrucutra del titulo
    \begin{abstract}
    Resumen del artículo.
\end{abstract}  
    \renewcommand{\tablename}{Tabla} %Para cambiar nombre: cuadro por tabla

    % Secciones del artículo 
    \section{Título de la primera sección}
Contenido de la primer sección. Primer bibliografia:\cite{bib1}


\subsection{Título de la primera subsección}
Contenido de la primer subsección. 

\begin{center}
Algo centrado\footnote{nota al pie} %nota al pie.
\end{center}


\subsection{Título de la segunda subsección}
Contenido de la segunda subsección. La tabla 1 es un ejemplo de tabla.

\begin{table} [h]%ambiente para tablas.
    \begin{center}
        \begin{tabular}{ccc} %iconstruccion de tabla, centrado, centrado, centrado.
            \textbf{UNO} & \textbf{DOS} & \textbf{TRES} \\ \hline %textbf negritas, \\salto de linea.
            a & b & c \\
            d & e & f \\ \hline %crea horizontal line.
        \end{tabular}
    \end{center}
    \caption{Ejemplo de tabla. \label{tabla}}
\end{table}
\pagebreak
\subsection{Título de la tercera subsección}
Contenido de la segunda subsección. A continuación se muestra un ejemplo de enumeración:

\begin{enumerate}
    \item \textit{uno.}
    \item \underline{dos.}
    \item \textbf{tres.}
    \item cuatro.
\end{enumerate}

    \section{Título de la Segunda sección}
Contenido de la segunda sección. Primera bibliografia:\cite{bib2}

 \subsection{Título de la primera subsección}
Contenido de la primer subsección. La figura 1 es un ejemplo de una Figura \ref{imagen2}.

\begin{figure}[h] %Ambiente para figuras, here.
    \begin{center}
        \includegraphics[scale=0.68]{secciones/imagenes/imagen2}
    \end{center}
    \caption{Ejemplo de figura. \label{imagen2}}
\end{figure}

\subsection{Título de la segunda subsección}
Contenido de la segunda subsección. Ejemplo de ecuación matemática es la ecuación 1:

\begin{equation}
    e=mc^2
    \label{ecuacion}
\end{equation}
Así se escribe la chicharronera: $x = \frac {-b \pm \sqrt {b^2 - 4ac}}{2a}$ 

\subsection{Título de la tercera subsección}
Contenido de la tercera subsección. A continuación se muestra un ejemplo de elementos en viñetas:

\begin{itemize} %ambiente o entorno itemize
    \item [+]\textit{uno.}
    \item \textbf{dos.}
    \item \underline{tres.}
    \item cuatro.
\end{itemize}



    %\input{}

    % Referencias
    \begin{thebibliography}{XXX0000}
    \bibitem{bib1} Autores. Título. País, Editorial y Año.
    \bibitem{bib2} Autores. Título. País, Editorial y Año.
\end{thebibliography}
\end{document}